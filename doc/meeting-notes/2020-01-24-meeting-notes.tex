\begin{Minutes}{Meeting with Claudia}
%%\subtitle{}
%%\moderation{}
%%\minutetaker{}
\participant{Stephen Malina, Claudia Shi}
%%\missingExcused{}
%%\missingNoExcuse{}
\minutesdate{January 21, 2020}
%%\starttime{}
%%\endtime{}
%%\cc{}
\maketitle
\topic{Initial approach problems and assumptions}
We started out with me describing the problem with our initial approach. I'd summarize this as, "we didn't account for the fact that mutation impacts binding and accessibility through the sequence mediator."

Then, as I was describing the problem, we realized that, if we have the following graph, then our original approach reflects an implicit assumption of a uniform distribution over sequences.
% Simple Bayesian network
\begin{figure}[ht]
  \begin{center}
      \begin{tikzpicture}

  % Define nodes
  \node[obs]                               (y) {$A$};
  \node[obs, left=of y, xshift=-1.2cm] (x) {$B$};
  \node[obs, left=of x] (s) {$S$};
  \node[obs, left=of s] (m) {$M$};
  \node[obs, above=of m] (r) {$R$};
  \node[above=of x, xshift=+1.4cm, latent] (u) {$U$};

  % Connect the nodes
  \edge {x, u} {y} ; %
  \edge {u, s} {x}
  \edge {m, r} {s}

\end{tikzpicture}

  \end{center}
  \caption{Graph of the causal DAG we get if we assume that mutation is upstream of sequence. Note that $ M $ represents mutation, $ S $ sequence, $ B $ binding, $ R $ randomization, and $ A $ accessibility.}
\end{figure}

Our data generation process using our neural net (in the binding case) gives us samples of the conditional distribution \( \Pr(b \mid \text{do}(m), s) \) and we want to estimate \( \Pr(B \mid \text{do}(M)) \). Basic probability algebra gives us
\[ \Pr(B \mid \text{do}(M)) = \int \Pr(B \mid \text{do}(M), S = s) \Pr(S = s) ds. \]
My understanding is that if we assume $ S $ is uniformly distributed, we can ignore the prior probability of $ S $ piece and then we get
\[ \Pr(B \mid \text{do}(M)) = \int \Pr(B \mid \text{do}(M), S=s) ds, \]
which we can approximate by sampling from the distribution of $ \Pr(B \mid S) $.

Assuming a uniform distribution over $ S $ seems like a strong assumption though.

\topic{Multi-DAG aggregation approach}
Then we discussed an alternate that David had hinted at in which we treat each sequence as instantiating its own DAG and then try to aggregate the causal effects from each. The benefit of this approach is that it's conceptually clearer. The downside is that it's still not clear to me that it makes to only have one sample from each DAG. On the other hand, if we try to get multiple samples from each DAG, we run into the problem of there being no obvious way to impose a dimension on the space of possible mutations but it seems like IV methods typically assume there's one.

\topic{Deep IV approach}
Finally, we discussed the Deep IV approach and Claudia mentioned that a bunch of people in Dave's lab are working on things related to Deep IV, including Victor.

\task[done]{Stephen}{Look into approaches to aggregating DAGs such as IPTW, AIPPW, etc.}

\end{Minutes}
