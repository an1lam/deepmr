%%%%%%%% ICML 2020 EXAMPLE LATEX SUBMISSION FILE %%%%%%%%%%%%%%%%%

\documentclass{article}

% Recommended, but optional, packages for figures and better typesetting:
\usepackage{microtype}
\usepackage{graphicx}
\usepackage{subfigure}
\usepackage{booktabs} % for professional tables

% hyperref makes hyperlinks in the resulting PDF.
% If your build breaks (sometimes temporarily if a hyperlink spans a page)
% please comment out the following usepackage line and replace
% \usepackage{icml2020} with \usepackage[nohyperref]{icml2020} above.
\usepackage{hyperref}

% Attempt to make hyperref and algorithmic work together better:
\newcommand{\theHalgorithm}{\arabic{algorithm}}

% Use the following line for the initial blind version submitted for review:
\usepackage{icml2020}

% If accepted, instead use the following line for the camera-ready submission:
%\usepackage[accepted]{icml2020}

% Packages I added:
\usepackage{todo}
% \presetkeys%
%     {todonotes}%
%     {inline,backgroundcolor=yellow}{}
\usepackage{amsmath}
\usepackage{amsthm}

\newtheorem{defn}{Definition}

% The \icmltitle you define below is probably too long as a header.
% Therefore, a short form for the running title is supplied here:
\icmltitlerunning{Deep MR}

\begin{document}

\twocolumn[
\icmltitle{Deep Mendelian Randomization: Using Mendelian Randomization to Detect Learned Causal Relationships in Deep Learning Models}

\todo*{Remove}
\nocite{langley00}
% It is OKAY to include author information, even for blind
% submissions: the style file will automatically remove it for you
% unless you've provided the [accepted] option to the icml2020
% package.

% List of affiliations: The first argument should be a (short)
% identifier you will use later to specify author affiliations
% Academic affiliations should list Department, University, City, Region, Country
% Industry affiliations should list Company, City, Region, Country

% You can specify symbols, otherwise they are numbered in order.
% Ideally, you should not use this facility. Affiliations will be numbered
% in order of appearance and this is the preferred way.
\icmlsetsymbol{equal}{*}

\begin{icmlauthorlist}
\icmlauthor{Stephen Malina}{equal,cu}
\icmlauthor{Daniel Cizin}{equal,cu}
\icmlauthor{David Knowles}{equal,cu,nygc}
\end{icmlauthorlist}
\icmlcorrespondingauthor{Stephen Malina}{sdm2181@columbia.edu}

\icmlaffiliation{to}{Department of Computer Science, Columbia University, New York, NY}
\icmlaffiliation{nygc}{New York Genome Center, New York, NY}

% You may provide any keywords that you
% find helpful for describing your paper; these are used to populate
% the "keywords" metadata in the PDF but will not be shown in the document
\icmlkeywords{Mendelian Randomization, Deep Learning, Causal Inference}

\vskip 0.3in
]

% this must go after the closing bracket ] following \twocolumn[ ...

% This command actually creates the footnote in the first column
% listing the affiliations and the copyright notice.
% The command takes one argument, which is text to display at the start of the footnote.
% The \icmlEqualContribution command is standard text for equal contribution.
% Remove it (just {}) if you do not need this facility.

%\printAffiliationsAndNotice{}  % leave blank if no need to mention equal contribution
\printAffiliationsAndNotice{\icmlEqualContribution} % otherwise use the standard text.

\begin{abstract}
\end{abstract}

\section{Introduction}
\label{introduction}

\subsection{Background \& Related Work}
\subsection{Mendelian Randomization}
Connect MR to IVs.
Briefly describe IVs and assumptions. Link to more comprehensive references.
Introduce idea of using multiple instruments to weaken assumptions. State assumptions Egger allows to relax and what it adds.

\subsection{Monte Carlo (MC) Dropout}
Need this to estimate uncertainty to feed to Mendelian randomization.
Allows us to get predictive means and standard errors for our predictions.

\section{Methods}
\subsection{Algorithm Overview}
Introduce notation that you'll use -- \textit{exposure}, \textit{outcome}, \textit{sequence}.

Describe input / output. 
Input: test set of one-hot encoded sequences for each exposure, model that predicts probability of binary exposure and outcome (also works for regression). 
Output: causal effect estimates for individual sequences and each exposure-outcome pair.

Recall that our goal is to determine whether data our model generates reflects known causal relationships. To accomplish this (see \todo*{figure 1} for visual depiction), we do the following for each transcription factor:
\begin{itemize}
    \item Randomly sample sequences for which experiments showing transcription factor binding (``reference sequences'').
    \item Perform \textit{saturation in-silico mutagenesis} for each reference sequence to generate \( \text{sequence\ length} \times \text{number\ of\ nucleotides} - 1 \) mutated sequences per original sequence.
    \item For each reference and set of mutated sequences, use MC-dropout \todo*{cite} to generate predictive means and standard errors of binding probabilities for the (reference|mutated) sequences.
    \item Generate \( \text{sequence length} \times \text{number of nucleotides} - 1 \) \textit{effect sizes} by taking the differences between each mutated sequence's predictive mean and the corresponding reference sequence's predictive mean and the standard error of these differences.
    \item Apply Mendelian randomization to each reference sequence's effect sizes and their standard errors to estimate a per-transcription factor, per-sequence region causal effect.
    \item \todo*{remove if we don't do it} Estimate overall per-transcription factor causal effects.
\end{itemize}

This leaves us with estimates of local (transcription factor and sequence level) and global (transcription factor level) causal effects. 

\subsection{Saturation In-Silico Mutagenesis}


\section{Experiments \& Results}

\subsection{Dataset \& Model}
\paragraph{Model.}
We use a pre-trained version of DeepSEA \todo*{cite} provided by the Kipoi \todo*{cite} library to generate binding and chromatin accessibility predictions.

\paragraph{Dataset.}
To generate predictions from DeepSEA, we use sequence regions from DeepSEA's held-out test set that had binding in ChIP-seq experiments for the relevant transcription factors. This data is provided as part of the ENCODE project. 

\subsection{DeepSEA transcription factor \& chromatin accessibility experimetn}
Ran Deep MR on DeepSEA's HepG2 TFs paired with chromatin accessibility.
Only use sequences where binding occurred to make linear effect more plausible.

Figure X depicts the distribution of causal effects of each of the N TFs on chromatin accessibility.

As you can see, Deep MR predicts positive causal effects for the N TFs on which we ran it, occasionally finding quite large causal effects.

Deep dive into causal effects for FOXA1. 

\section{Discussion}
\subsection{MR finds uniformly positive causal effects}

\subsection{MR finds significant causal effects with minimal pleiotropy}



\subsection{Limitations}
\paragraph{Mendelian randomization assumptions.}

\begin{itemize}
    \item MR assumptions brief sentence
\end{itemize}

\section*{Software and Data}

\bibliography{example_paper}
\bibliographystyle{icml2020}
\end{document}


% This document was modified from the file originally made available by
% Pat Langley and Andrea Danyluk for ICML-2K. This version was created
% by Iain Murray in 2018, and modified by Alexandre Bouchard in
% 2019 and 2020. Previous contributors include Dan Roy, Lise Getoor and Tobias
% Scheffer, which was slightly modified from the 2010 version by
% Thorsten Joachims & Johannes Fuernkranz, slightly modified from the
% 2009 version by Kiri Wagstaff and Sam Roweis's 2008 version, which is
% slightly modified from Prasad Tadepalli's 2007 version which is a
% lightly changed version of the previous year's version by Andrew
% Moore, which was in turn edited from those of Kristian Kersting and
% Codrina Lauth. Alex Smola contributed to the algorithmic style files.
