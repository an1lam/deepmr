\begin{Minutes}{}
%%\subtitle{}
%%\moderation{}
%%\minutetaker{}
\participant{Stephen Malina}
%%\missingExcused{}
%%\missingNoExcuse{}
\minutesdate{March 12, 2020}
%%\starttime{}
%%\endtime{}
%%\cc{}
\maketitle
\topic{Current Status}
Currently working on comparing sequence snippets in which mutations disrupt FOXA1 binding \textit{in silico} to FOXA1's true motifs. I currently have code to print out the sequence context for the most impactful mutations. Before I go and look for motifs, it probably makes sense to try and do some basic aggregation to see whether any motifs pop up more often than others. I have an idea for how to do this that involves looking at all \( k \)-mers for \( k \in [3, 5] \) and counting appearances.

\topic{Defining a motif similarity threshold}
Now that I've written the code to count appearance of k-mers around impactful mutations, I'm realizing that I need a better criteria for success than "looks like some of the motifs we find are significant." In the spirit of coming up with my own ideas before looking into existing literature, I'm going to try and frame the problem in my own words and see if I can come up with a tentative solution before looking up others'. 

A starting point is just asking whether we could validate the hypothesis that the disruptions reflect \textbf{some} underlying pattern. We can frame this as a hypothesis testing problem where we want to show that the k-mer counts meaningfully differ from what we'd expect under a uniform model. I'm not going to actually go through the math but I'm confident this is doable. 

Writing the above out does give me an idea. I was previously thinking of this from the perspective of trying to figure out how to validate comparisons between in silico and observed motifs, but maybe that's the wrong place to start. Instead, maybe we can use the above hypothesis testing approach to find k-mers whose counts have a low probability under a uniform (or normal) counts model. If we set \( \alpha \) for choosing a probability threshold (will be at both tails) low enough, that should leave us with few enough k-mers that we can then hopefully just use edit distance and some (different threshold) to compare to observed motifs. This doesn't deal with the question of \textit{how} to decide how similar is similar enough but it does deal with the issue of having too many k-mers to deal with such that it's likely that at least a few will match the known FOXA1 motifs.

\end{Minutes}
