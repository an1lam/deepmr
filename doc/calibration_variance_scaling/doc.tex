\documentclass[12pt]{article}

% fonts

\usepackage[T1]{fontenc}
\usepackage[full]{textcomp}
\usepackage{newtxtext}
\usepackage{cabin} % sans serif
\usepackage[varqu,varl]{inconsolata} % sans serif typewriter
\usepackage[final,expansion=alltext]{microtype}
\usepackage[english]{babel}
\usepackage{amsmath}
\usepackage[bigdelims,vvarbb]{newtxmath} % bb from STIX
\usepackage[cal=boondoxo]{mathalfa} % mathcal

% geometry of the page

\usepackage[top=1in,
            bottom=1in,
            left=1in,
            right=1in]{geometry}

% paragraph spacing

\setlength{\parindent}{0pt}
\setlength{\parskip}{2ex plus 0.4ex minus 0.2ex}

% useful packages

\usepackage{natbib}
\usepackage{epsfig}
\usepackage{url}
\usepackage{bm}
\usepackage{blindtext}


\begin{document}

\begin{flushleft}
\textbf{Deriving the calibration scaling term for a deep ensemble} \\
Stephen Malina \\
\today
\end{flushleft}

\vspace{0.1in}

\normalsize

Assume that we've trained \( m \) regression models, denoted \( f_i \) for \( i \in [m] \), with different random seeds. Let \( \mu(x_i) \) denote
the ensemble mean,
\[ \frac{1}{m} \sum_{i=1}^m f_i(x_i), \]
and \( \sigma^2(x_i) \) the predictive variance,
\[ \frac{1}{m} \sum_{i=1}^m (f_i(x_i) - \mu(x_i))^2, \]
for a single data point.

As is often done for ensembles, suppose that
\[ y_i \sim \mathcal{N}(\mu(x_i), \lambda \cdot \sigma^2(x_i). \]

We'll now derive a formula for \( \lambda \) that maximizes the \( \log \)-likelihood of \( \mathbf{y} \). 

The \( \log \)-likelihood is
\begin{align}
	\mathcal{L} &= \sum_{i=1}^n \log p(y_i \mid \mu(x_i), \lambda \sigma^2(x_i)) \\
		    &= -\frac{n}{2} \log(2\pi) - \frac{n}{2} \log \lambda - \sum_{i=1}^n \log(\sigma(x_i)) - \sum_{i=1}^n \left[ \frac{1}{2 * \lambda * \sigma^2(x_i)} \left( y_i - \mu(x_i) \right)^2 \right].
\end{align}
Its derivative with respect to \( \lambda \) is
\begin{equation}
	\frac{\partial \mathcal{L}}{\partial \lambda} = -\frac{n}{2\lambda} + \sum_{i=1}^n \frac{1}{2 * \lambda^2 * \sigma^2(x_i)} (y_i - \mu(x_i))^2.
\end{equation}

Setting this to \( 0 \) and solving gives us
\begin{align}
	& -\frac{n}{2\lambda} + \sum_{i=1}^n \frac{1}{2 * \lambda^2 * \sigma^2(x_i)} (y_i - \mu(x_i))^2 = 0 \\
	& \sum_{i=1}^n \frac{1}{2 * \lambda^2 * \sigma^2(x_i)} (y_i - \mu(x_i))^2 = \frac{n}{2 \lambda} \\
	& \frac{1}{n} \sum_{i=1}^n \frac{1}{\sigma^2(x_i)} (y_i - \mu(x_i))^2 = \lambda.
\end{align}

Intuitively, this means that \( \lambda \) maximizes the \( \log \)-likelihood when it's set to the average variance-weighted mean-squared error.


\end{document}

%%% Local Variables:
%%% mode: latex
%%% TeX-master: t
%%% End:
